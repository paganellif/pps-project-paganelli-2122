\documentclass{report}
\usepackage[italian]{babel}
\usepackage[dvipsnames]{xcolor}
\usepackage[utf8]{inputenc}
\usepackage{hyperref}
\usepackage{multicol}
\usepackage{setspace}
\usepackage{fancyhdr}
\usepackage{mathtools}
\DeclarePairedDelimiter\ceil{\lceil}{\rceil}
\DeclarePairedDelimiter\floor{\lfloor}{\rfloor}
\usepackage{url}
\usepackage{graphicx}
\linespread{1.2}
\usepackage{float}
\usepackage{pgfplots}
\usepackage{listings}
\usepackage{indentfirst}
\pgfplotsset{width=10cm,compat=1.9}
\usepgfplotslibrary{external}
\tikzexternalize

\newcommand{\emailaddr}[1]{\href{mailto:#1}{\texttt{#1}}}
\usepackage{subfiles} % Best loaded last in the preamble

\lstset{basicstyle=\ttfamily,
  showstringspaces=false,
  commentstyle=\color{red},
  keywordstyle=\color{blue},
  inputencoding=utf8,
  extendedchars=true
}

\usepackage{color}
\definecolor{lightgray}{rgb}{.9,.9,.9}
\definecolor{darkgray}{rgb}{.4,.4,.4}
\definecolor{purple}{rgb}{0.65, 0.12, 0.82}

\lstdefinelanguage{JavaScript}{
  keywords={typeof, new, true, false, catch, function, return, null, catch, switch, var, if, it, in, while, do, else, case, object, type, break, implicit, class, def, yield, val, for},
  keywordstyle=\color{blue}\bfseries,
  ndkeywords={export, boolean, throw, implements, import, this},
  ndkeywordstyle=\color{darkgray}\bfseries,
  identifierstyle=\color{black},
  sensitive=false,
  comment=[l]{//},
  morecomment=[s]{/*}{*/},
  commentstyle=\color{purple}\ttfamily,
  stringstyle=\color{red}\ttfamily,
  morestring=[b]',
  morestring=[b]"
}

\lstset{
   language=JavaScript,
   backgroundcolor=\color{lightgray},
   extendedchars=true,
   basicstyle=\footnotesize\ttfamily,
   showstringspaces=false,
   showspaces=false,
   numbers=left,
   numberstyle=\footnotesize,
   numbersep=9pt,
   tabsize=2,
   breaklines=true,
   showtabs=false,
   captionpos=b,
   columns=fullflexible,
   postbreak=\mbox{\textcolor{red}{$\hookrightarrow$}\space},
}

\title{\LARGE \textbf{Paradigmi di Programmazione e Sviluppo} \\ FRP-Scala: Studio e sperimentazione del paradigma Functional Reactive Programming in Scala
}

\author{
    Filippo Paganelli \\ \emailaddr{filippo.paganelli3@studio.unibo.it}
    \\ matricola 0000926989 \\ A.A 21/22
}

\date{Gennaio 2022}

\usepackage{ifxetex,ifluatex}

\ifxetex
  \usepackage{catchfile}
  \newcommand\getenv[2][]{%
    \immediate\write18{kpsewhich --var-value #2 > \jobname.tmp}%
    \CatchFileDef{\temp}{\jobname.tmp}{\endlinechar=-1}%
    \if\relax\detokenize{#1}\relax\temp\else\let#1\temp\fi}
\else
  \ifluatex
    \newcommand\getenv[2][]{%
      \edef\temp{\directlua{tex.sprint(
        kpse.var_value("\luatexluaescapestring{#2}") or "" ) }}%
      \if\relax\detokenize{#1}\relax\temp\else\let#1\temp\fi}
  \else
    \usepackage{catchfile}
    \newcommand{\getenv}[2][]{%
      \CatchFileEdef{\temp}{"|kpsewhich --var-value #2"}{\endlinechar=-1}%
      \if\relax\detokenize{#1}\relax\temp\else\let#1\temp\fi}
  \fi
\fi



\begin{document}

\maketitle
\newpage
\begin{abstract}
Lo scopo del progetto individuale consiste nello studio e nella sperimentazione del paradigma Functional Reactive Programming (FRP) in Scala. Il primo capitolo tratta lo studio delle principali caratteristiche del paradigma FRP, derivanti dall'unione dei due paradigmi funzionale (FP) e reattivo (RP), e della architettura di una applicazione FRP. Successivamente vengono analizzate e confrontate alcune librerie open-source per il paradigma FRP in Scala, scelte fra quelle disponibili al momento dello svolgimento del progetto, sviluppando alcuni semplici esempi di test per sperimentare pro e contro delle librerie. Nell'ultimo capitolo si descrive come è stata progettata e sviluppata una semplice rivisitazione del videogioco Snake utilizzando il paradigma FRP, il linguaggio Scala e una fra le librerie individuate nel capitolo precedente.

Tutto il codice, esempi e applicazione, è disponibile al repository pubblico Github \url{https://github.com/paganellif/pps-project-paganelli-2122}.
% TODO: completare introduzione indicando le librerie individuate/analizzate nella seconda sezione
% TODO: completare introduzione indicando la libreria scelta, quale mini-applicazione è stata fatta
\end{abstract}
\newpage
\tableofcontents

%\newpage
%\chapter{Introduzione}\label{introduzione}
%\subfile{chapters/introduzione-1}

\newpage
\chapter{Functional Reactive Programming}\label{frp}
\subfile{chapters/frp-1}

\newpage
\chapter{Analisi Librerie Scala}\label{analisi}
\subfile{chapters/analisi-2}

\newpage
\chapter{Mini-App: FRP Snake}\label{mini-app}
\subfile{chapters/mini-app-3}

\newpage
\chapter{Conclusioni}\label{conclusioni}
\subfile{chapters/conclusioni-4}

\newpage
%\addcontentsline{toc}{section}{Riferimenti bibliografici}
\begin{thebibliography}{4}

\bibitem{frp1}
Rambabu Posa, \\ \textit{Scala Reactive Programming}, \\ Packt, 2018

\bibitem{frp2}
\textit{Functional Reactive Programming}, \\
\url{https://en.wikipedia.org/wiki/Functional_reactive_programming}

\bibitem{frp3}
Jonas Bonér, Dave Farley, Roland Kuhn and Martin Thompson, \\ \textit{The Reactive Manifesto}, \\
  \url{https://www.reactivemanifesto.org/}

\bibitem{frp4}
Reactive Streams Special Interest Group, \\ \textit{Reactive Streams}, \\
  \url{https://www.reactive-streams.org/}

\bibitem{frp5}
Wan Zhanyong and Hudak Paul, \\ \textit{Functional Reactive Programming from First Principles}, \\
Association for Computing Machinery, 2000

\bibitem{frp6}
Conal Elliott, \\ \textit{Push-pull functional Reactive Programming}, \\
\url{http://conal.net/papers/push-pull-frp}, 2009

\bibitem{frp7}
Stephen Blackheath, Anthony Jones, \\ \textit{Functional Reactive Programming}, \\
Manning, 2016

\end{thebibliography}

\end{document}
