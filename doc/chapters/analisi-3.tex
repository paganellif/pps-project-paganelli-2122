\documentclass[../main.tex]{subfiles}

\begin{document}
In questa sezione si analizzano alcune delle librerie scelte per la sperimentazione del paradigma FRP in Scala.

\section{Confronto Librerie}

\subsection{Sodium}
La libreria open-source Sodium è stata progettata e sviluppata da Stephen Blackheath e Anthony Jones (più altri collaboratori) per fornire una libreria production-ready in diversi linguaggi, tra cui Scala, per promuovere la vera definizione del paradigma FRP e per realizzare un riferimento/benchmark per future librerie.

\begin{table}[H]
\centering
\begin{tabular}{|c|c|}
     \hline
     Repository & https://github.com/SodiumFRP/sodium \\
     \hline
     Versione Latest Release & 1.2.0 \\
     \hline
     Data Latest Release & Ottobre 2019 \\
     \hline
     Data Latest Commit & 3 Febbraio 2020 \\
     \hline
     Supporto Scala & 2.12 \\
     \hline
     Dipendenza Gradle & nz.sodium:sodium:1.2.0 \\
     \hline
\end{tabular}
\caption{Main Info Libreria Sodium}
\end{table}

\subsection{Reactify}
Reactify è una libreria open-source che permette di sviluppare sistemi col paradigma FRP fornendo un insieme limitato di concetti rispetto ad altre librerie come Sodium. Il programma viene espresso in termini di variabili che cambiano (\textit{Var}) o non cambiano (\textit{Val}) valore nel tempo e nel definire una reazione al cambiamento: tutte le primitive tipiche del paradigma FRP non sono modellate la libreria, la quale permette di usare qualsiasi funzionalità di Scala direttamente.

\begin{table}[H]
\centering
\begin{tabular}{|c|c|}
     \hline
     Repository & https://github.com/outr/reactify \\
     \hline
     Versione Latest Release & 4.0.6 \\
     \hline
     Data Latest Release & Maggio 2021 \\
     \hline
     Data Latest Commit & 17 Maggio 2021 \\
     \hline
     Supporto Scala & 2.11, 2.12, 2.13, 3 \\
     \hline
     Dipendenza Gradle & com.outr:reactify_2.13:4.0.6 \\
     \hline
\end{tabular}
\caption{Main Info Libreria Reactify}
\end{table}

\end{document}

