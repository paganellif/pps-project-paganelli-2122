\documentclass[../main.tex]{subfiles}

\begin{document}
In questa sezione si analizzano alcune delle librerie scelte per la sperimentazione del paradigma FRP in Scala.
\section{Confronto Librerie}
\subsection{Akka Stream}
Akka Stream è un modulo del Toolkit Akka Lightbend che permette la manipolazione di stream di dati asincroni e con backpressure non-bloccante seguendo le specifiche Reactive Stream \cite{frp4} e Reactive Manifesto \cite{frp3}.\\
La libreria è stata aggiunta al progetto come dipendenza Gradle:
\begin{lstlisting}[basicstyle=\small,caption={Dipendenze Akka Stream Gradle},captionpos=b,frame=single]
implementation platform("com.typesafe.akka:akka-bom_2.13:2.6.17")
implementation "com.typesafe.akka:akka-stream_2.13"
\end{lstlisting}

I principali componenti della libreria sono:
\begin{description}
  \item[Source] -
  \item[Sink] - 
  \item[Flow] - 
  \item[RunnableGraph] - 
  \item[Fan-in Function] -
  \item[Fan-out Function] -
\end{description}
\subsection{Sodium}
\section{Casi d'uso}
\end{document}

