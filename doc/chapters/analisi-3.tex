\documentclass[../main.tex]{subfiles}

\begin{document}
In questa sezione si analizzano alcune delle librerie scelte per la sperimentazione del paradigma FRP in Scala.

\section{Confronto Librerie}

\subsection{Akka Stream}
Akka Stream è un modulo del Toolkit Akka Lightbend che permette la manipolazione di stream di dati asincroni e con backpressure non-bloccante seguendo le specifiche Reactive Stream \cite{frp4} e Reactive Manifesto \cite{frp3}.

\begin{table}[H]
\centering
\begin{tabular}{|c|c|}
     \hline
     Repository & https://github.com/akka/akka \\
     \hline
     Versione Latest Release & 2.16.17 \\
     \hline
     Data Latest Release & Ottobre 2021 \\
     \hline
     Supporto Scala & 2.21, 2.13, 3 \\
     \hline
     Dipendenza Gradle & com.typesafe.akka:akka-stream\_2.13 \\
     \hline
\end{tabular}
\caption{Main Info Libreria Akka Stream}
\end{table}


I principali componenti della libreria sono:
\begin{description}
  \item[Source] -
  \item[Sink] - 
  \item[Flow] - 
  \item[RunnableGraph] - 
  \item[Fan-in Function] -
  \item[Fan-out Function] -
\end{description}


\subsection{Sodium}
La libreria open-source Sodium è stata progettata e sviluppata da Stephen Blackheath e Anthony Jones (più altri collaboratori) per fornire una libreria production-ready in diversi linguaggi, tra cui Scala, per promuovere la vera definizione del paradigma FRP e per realizzare un riferimento/benchmark per future librerie.

\begin{table}[H]
\centering
\begin{tabular}{|c|c|}
     \hline
     Repository & https://github.com/SodiumFRP/sodium \\
     \hline
     Versione Latest Release & 1.2.0 \\
     \hline
     Data Latest Release & Ottobre 2019 \\
     \hline
     Supporto Scala & 2.12 \\
     \hline
     Dipendenza Gradle & nz.sodium:sodium:1.2.0 \\
     \hline
\end{tabular}
\caption{Main Info Libreria Sodium}
\end{table}
\section{Casi d'uso}
\end{document}

