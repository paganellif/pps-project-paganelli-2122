\documentclass[../main.tex]{subfiles}

\begin{document}
Lo scopo del progetto individuale consiste nello studio e nella sperimentazione del paradigma Functional Reactive Programming (FRP) in Scala. La prima sezione riguarda lo studio delle principali caratteristiche del paradigma FRP, derivanti dall'unione dei paradigmi funzionale (FP) e reattivo (RP). Successivamente vengono analizzate e confrontate le possibili librerie per il paradigma FRP in Scala, sviluppando alcuni semplici esempi. Infine si descrive una mini-applicazione realizzata per sperimentare il paradigma FRP scegliendo una delle librerie individuate nella precedente sezione. Tutto il codice, esempi e mini-applicazione, è disponibile al repository pubblico Github \url{https://github.com/paganellif/pps-project-paganelli-2122}.
% TODO: completare introduzione indicando le librerie individuate/analizzate nella seconda sezione
% TODO: completare introduzione indicando la libreria scelta, quale mini-applicazione è stata fatta

\end{document}