\documentclass[../main.tex]{subfiles}

\begin{document}
% dire che c'è abbastanza confusione in letteratura per comprendere le differenze tra il paradigma frp e i reactive stream/programmazione reattiva base
% dire che per scala non ci sono tool maturi/diffusi per il paradigma FRP e che i pochi tool "maturi" sono per altri linguaggi come haskell
% il paradigma frp con la libreria sodium scelta (nonostante sia stata adattata per usarla nativamente in scala 2.13.7) ha comunque permesso di realizzare una rivisitazione di snake funzionante --> sicuramente approfondirei ulteriormente questo paradigma combinato con scala e lo terrei in considerazione per lo sviluppo di futuri progetti
\end{document}