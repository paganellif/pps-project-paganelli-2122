\documentclass[../main.tex]{subfiles}

\begin{document}
Seppur maturo teoricamente come paradigma di programmazione, nella ricerca e nella analisi di librerie per lo sviluppo di applicazioni FRP in Scala si è scoperto che i maggiori player nel panorama open-source (ad esempio \textit{Lightbend} con gli \textit{Akka Stream} o \textit{Typelevel} con \textit{FS2}) sono nella direzione dei cosi detti \textit{Reactive Streams}, i quali hanno alcuni concetti differenti dalla programmazione funzionale reattiva come visto nel primo capitolo.

Tra le poche alternative individuate sono state analizzate e confrontate le due librerie FRP per Scala: \textit{Sodium} e \textit{Reactify}. La prima libreria a differenza della seconda fornisce una API che permette una rappresentazione più semplice e leggibile del grafo delle dipendenze e per questo è stata adottata per lo sviluppo di una applicazione FRP d'esempio.

I concetti alla base del paradigma FRP e la libreria \textit{Sodium} mi hanno permesso di progettare e implementare la applicazione d'esempio in minor tempo rispetto a quello atteso. Modellato il dominio della applicazione e progettato correttamente il grafo delle dipendenze mappando gli input con gli output è stata semplice e veloce la conversione in codice Scala. Sicuramente approfondirei ulteriormente questo paradigma per considerarlo nello sviluppo di progetti futuri, sia in ambito universitario che lavorativo.
\end{document}