\documentclass[12pt]{article}

\usepackage[utf8]{inputenc}
%\usepackage[T1]{fontenc}

\usepackage{geometry}
\geometry{a4paper}
\usepackage{graphicx}
\usepackage{float}
\usepackage[italian]{babel}
\usepackage{url}
\DeclareUnicodeCharacter{2212}{-}
\linespread{1.2}
\setlength{\parindent}{0pt}
\usepackage{listings}
\usepackage{fancyhdr}
\usepackage{hyperref}
\usepackage{multicol}
\usepackage{setspace}

\pagestyle{fancy}
\fancyhf{}
\fancyhead[LE,RO]{\leftmark}
\fancyfoot[LE,RO]{\thepage}
\begin{document}

%----------------------------------------------------------------------------------------
%	TITOLO
%----------------------------------------------------------------------------------------

\begin{titlepage}

\newcommand{\HRule}{\rule{\linewidth}{0.5mm}}

\center

\textsc{\Large Relazione Paradigmi di Programmazione e Sviluppo}\\[0.5cm]

\HRule \\[0.4cm]
\setstretch{1.5}{ \huge \bfseries FRP-scala: Studio e sperimentazione del paradigma Functional Reactive Programming in Scala } \\
\HRule \\[0.4cm]
\center A.A. 2021/2022
\vfill

\begin{flushleft}
Filippo Paganelli \\ 0000926989 \\ filippo.paganelli3@studio.unibo.it
\end{flushleft}


\end{titlepage}

\tableofcontents

\newpage
\section{Introduzione}

\newpage
\addcontentsline{toc}{section}{Riferimenti bibliografici}
\begin{thebibliography}{4}

%\bibitem{ws-primer}
%Grigorious Antoniou, Frank van Harmelen, \\
%\textit{A Semantic Web Primer - second edition}, \\ The MIT Press

%\bibitem{neo4j2}
%Bryce Merkl Sasaki, \\
%\href{https://neo4j.com/blog/other-graph-database-technologies/?ref=blog}{\textit{Graph Databases for Beginners: Other Graph Technologies}}, \\ Neo4J

%https://rdflib.readthedocs.io/en/stable/univrdfstore.html
%https://stackoverflow.com/questions/tagged/rdflib

\end{thebibliography}
%----------------------------------------------------------------------------------------

\end{document}